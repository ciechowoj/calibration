
\newcommand{\ds}[0]{\displaystyle}
\newcommand{\ml}[0]{\mathlarger}

\newcommand{\smatheq}[1] {
    \scalebox{0.75}{
        \begin{minipage}{1.2\linewidth}
            \begin{align*}
                #1
            \end{align*}
        \end{minipage}
    }
}

\newcommand{\matheq}[1] {
$$
\begin{aligned}
#1
\end{aligned}
$$
}

\newcommand{\nmatheq}[1] {
\begin{equation}
\begin{aligned}
#1
\end{aligned}
\end{equation}
}

\newcommand{\lmatheq}[1] {
    \scalebox{1.3}{
        \begin{minipage}{0.744\linewidth}
            \begin{align*}
                #1
            \end{align*}
        \end{minipage}
    }
}

\newcommand{\bmatheq}[1] {
    \scalebox{1.0}{
        \begin{minipage}{0.98\linewidth}
            \begin{align*}
                #1
            \end{align*}
        \end{minipage}
    }
}

\newcommand{\hmatheq}[1] {
\[ \scalebox{2}{$#1$} \]
}

\newcommand{\code}[1]{\texttt {\noindent\ignorespaces #1}}

\newcommand{\blist}[1] {
\begin{itemize}
#1
\end{itemize}
}

\newcommand{\rarrow}{\rightarrow}
\newcommand{\larrow}{\leftarrow}
\newcommand{\lrarrow}{\leftrightarrow}
\newcommand{\norm}[1]{\left\lVert#1\right\rVert}

\newcommand{\xa}[1] {\left< #1 \right> }
\newcommand{\xb}[1] {\left{ #1 \right} }
\newcommand{\xp}[1] {\left( #1 \right)}
\newcommand{\xs}[1] {\left[ #1 \right]}


\newcommand{\E}[1] {E \left[ #1 \right]}
\newcommand{\xE}[1] {E \left[ #1 \right]}
\newcommand{\xsum}[2] {\sum\limits_{#1}^{#2}}
\newcommand{\xint}[2] {\int_{#1}^{#2}}
\newcommand{\xd}[1] {\, d#1}

\newcommand{\omegai}[0] {\omega_i}
\newcommand{\omegao}[0] {\omega_o}

\newcommand{\red}[1] {\begingroup\color{red}#1\endgroup}

\makeatletter
    \newenvironment{withoutheadline}{
        \def\beamer@entrycode{\vspace*{-\headheight}}
    }{}
\makeatother


\newcommand{\imageframe}[2]{
    \begin{withoutheadline}
    \begin{frame}
        \centerline{\includegraphics[width=#2\textwidth]{#1}}
        \begin{flushright}
        [Qin 2015]
        \end{flushright}
    \end{frame}
    \end{withoutheadline}
}

\newcommand{\cev}[1]{\reflectbox{\ensuremath{\vec{\reflectbox{\ensuremath{#1}}}}}}

% http://tex.stackexchange.com/questions/114321/extensible-vec-instead-of-overrightarrow
\makeatletter
\newcommand{\lvec}[1]{%
    \vbox{\m@th \ialign {##\crcr
    \vectfill\crcr\noalign{\kern-\p@ \nointerlineskip}
    $\hfil\displaystyle{#1}\hfil$\crcr}}}
\def\vectfill{%
    $\m@th\smash-\mkern-7mu%
    \cleaders\hbox{$\mkern-2mu\smash-\mkern-2mu$}\hfill
    \mkern-7mu\raisebox{-4.45pt}[\p@][\p@]{$\mathord\mathchar"017E$}$}

\renewcommand{\lvec}{%
    \mathpalette {\overarrow@\vectfill@}}
\def\vectfill@{\arrowfill@\relbar\relbar{\raisebox{-4.45pt}[\p@][\p@]{$\mathord\mathchar"017E$}}}

\renewcommand{\lvec}{%
    \mathpalette {\overarrow@\vectfillb@}}
\newcommand{\vecbar}{%
    \scalebox{1}[0.86]{$\relbar$}}
\def\vectfillb@{\arrowfill@\vecbar\vecbar{\raisebox{-4.69pt}[\p@][\p@]{$\mathord\mathchar"017E$}}}
\makeatother



\newcommand{\hvector}[1]{
    \begin{bmatrix}
    #1
    \end{bmatrix}
}

\newcommand{\tvector}[1]{
    \begin{bmatrix}
    #1
    \end{bmatrix}^T
}

\renewcommand{\matrix}[1]{
    \begin{bmatrix}
    #1
    \end{bmatrix}
}
